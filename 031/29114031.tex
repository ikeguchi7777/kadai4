\documentclass{jarticle}

\usepackage[dvipdfmx]{graphicx}
\usepackage{url}
\usepackage{listings,jlisting}
\usepackage{ascmac}
\usepackage{amsmath,amssymb}

%ここからソースコードの表示に関する設定
\lstset{
    basicstyle={\ttfamily},
    identifierstyle={\small},
    commentstyle={\smallitshape},
    keywordstyle={\small\bfseries},
    ndkeywordstyle={\small},
    stringstyle={\small\ttfamily},
    frame={tb},
    breaklines=true,
    columns=[l]{fullflexible},
    numbers=left,
    xrightmargin=0zw,
    xleftmargin=3zw,
    numberstyle={\scriptsize},
    stepnumber=1,
    numbersep=1zw,
    lineskip=-0.5ex
}
%ここまでソースコードの表示に関する設定 

\title{知能プログラミング演習II 課題4}
\author{グループ07\\
    29114031 大原 拓人\\
%  {\small (グループレポートの場合は、グループ名および全員の学生番号と氏名が必要)}
}
\date{2019年11月19日}

\begin{document}
\maketitle

\paragraph{提出物} 個人レポート、グループプログラム「group07.zip」
\paragraph{グループ} グループ07
\paragraph{メンバー}
\begin{tabular}{|c|c|c|}
    \hline
    学生番号&氏名&貢献度比率\\
    \hline\hline
    29114007&池口弘尚&0\\
    \hline
    29114031&大原拓人&0\\
    \hline
    29114048&北原太一&0\\
    \hline
    29114086&飛世裕貴&0\\
    \hline
    29114095&野竹浩二朗&0\\
    \hline
\end{tabular}

\section{課題の説明}
\begin{description}
    \item[必須課題4-1] まず,教科書3.2.1の「前向き推論」のプログラムと教科書3.2.2の
    「後向き推論」のプログラムとの動作確認をし,前向き推論と後ろ向き推論の違いを説明せよ.
    また,実行例を示してルールが選択される過程を説明せよ.説明の際には,LibreOfficeの
    Draw(コマンド soffice --draw)などのドロー系ツールを使ってp.106 図3.11や
    p.118 図3.12のような図として示すことが望ましい.
    \item[必須課題4-2] CarShop.data , AnimalWorld.data 等のデータファイルを
    実際的な応用事例(自分達の興味分野で良い)に書き換えて,前向き推論,
    および後ろ向き推論に基づく質問応答システムを作成せよ.どのような応用事例を扱うかは,メンバーで話し合って決めること.
    なお,ユーザの質問は英語や日本語のような自然言語が望ましいが,難しければ変数を含むパターン等でも可とする.
    \item[必須課題4-3] 上記4-2で実装した質問応答システムのGUIを作成せよ.
    質問に答える際の推論過程を可視化できることが望ましい.
    \item[発展課題4-4] 上記4-3で実装したGUIを発展させ,質問応答だけでなく,
    ルールの編集(追加,削除,変更)などについてもGUIで行えるようにせよ.
\end{description}


\section{課題4-1}
\begin{screen}
    まず,教科書3.2.1の「前向き推論」のプログラムと教科書3.2.2の
    「後向き推論」のプログラムとの動作確認をし,前向き推論と後ろ向き推論の違いを説明せよ.
    また,実行例を示してルールが選択される過程を説明せよ.説明の際には,LibreOfficeの
    Draw(コマンド soffice --draw)などのドロー系ツールを使ってp.106 図3.11や
    p.118 図3.12のような図として示すことが望ましい.
\end{screen}
\subsection{手法}
    
\subsection{実装}
    
\subsection{考察}
    
\section{課題4-2}
\begin{screen}
    CarShop.data , AnimalWorld.data 等のデータファイルを
    実際的な応用事例(自分達の興味分野で良い)に書き換えて,前向き推論,
    および後ろ向き推論に基づく質問応答システムを作成せよ.どのような応用事例を扱うかは,メンバーで話し合って決めること.
    なお,ユーザの質問は英語や日本語のような自然言語が望ましいが,難しければ変数を含むパターン等でも可とする.
\end{screen}
    私は自然言語(英語)によって質問を解釈する質問応答システムを担当した。
\subsection{手法}
    OpenNLPのライブラリを利用し、入力した英文とデータベースの英文
    を文節に分けて、比較しやすいようにする。
\subsection{実装}
    まず、文節に分けられたことがわかるように、与えられた文章を文節で
    区切ったものを出力するgetChunklistメソッドをSentenceクラスに作成した。

\begin{lstlisting}[caption=Sentence.java]
    String getChunklist(){
		String result = "";
		for (Chunk chunk : this) {
			result+="".concat("/").concat(chunk.getMorphemes());
		}
		return result;
	}
\end{lstlisting}
    getMorphemesメソッドは以下のようになっている。
\begin{lstlisting}[caption=Chunk.java]
    public String getMorphemes(){
		String line = "";
		for (Morpheme word : Chunk.this) {
			line+="".concat(word.getWord()).concat(" ");
		}
        return line;
    }
\end{lstlisting}

\subsection{考察}
    

\section{感想}

% 参考文献
\begin{thebibliography}{99}
    \bibitem{opennlp} ホームページtutorialspoint内の「OpenNLP」
\end{thebibliography}

\end{document}